\documentclass[a4paper,10pt]{article}

\usepackage{palatino}
\usepackage{mathpazo}           % optional [sc] fuer real small caps
\usepackage[left=3cm,top=3cm,right=3cm,bottom=2cm,nohead,nofoot]{geometry}

\usepackage{graphicx}

\newcommand{\key}[1]{{{\texttt{\textbf{#1}}}}} % this does nasty things to underscores
\newcommand{\sex}[1]{{{\textbf{#1}}}} % \sec already defined

\newcommand{\crawl}{\textsc{Crawl}}
\newcommand{\dungeon}{\textsc{Dungeon}}

\newcommand{\spacecolumn}{\begin{minipage}[t]{2cm}\phantom{xxxx}\end{minipage}}
\newcommand{\para}{\vspace{1.5ex}}
\setlength{\parindent}{0em}

\newcommand{\mc}[1]{\multicolumn{2}{l}{#1}}

\newcounter{abccounter}
\newenvironment{abcliste}{\begin{list}{(\alph{abccounter})}
                      {\usecounter{abccounter}
                       \setlength{\topsep}{0ex}
                       \setlength{\partopsep}{0ex}
                       \setlength{\listparindent}{0ex}
                       \setlength{\itemsep}{0ex}
                       \setlength{\parsep}{0ex}
                       \setlength{\leftmargin}{2em}
                       \setlength{\labelwidth}{2em}
                       \setlength{\parskip}{0ex}
                      }
                      }{\end{list}}

\pagestyle{empty}

\begin{document}


\begin{center}\textbf{\LARGE
\dungeon\ \crawl: Files and Contact
}\end{center}

This is the reference sheet for the roguelike game \dungeon\ \crawl,
specifically for the current version of the \textsc{Stone Soup} branch. 
\crawl\ is a game of dungeon exploration, fighting and magic that is
renowned for its intricate skills and religion systems. Success requires
tactics, strategy, and perseverance. Though \crawl's reputation is 
close to devilish, victories were reported\dots

\para

This page explains the various important files. The next page lists a 
number of important changes introduced in version 0.4. The last two 
pages give a very brief introduction to the game, which should be 
enough to get you started. If you are completely new to this type of 
game and still want to plunge right in, start up a new game and select 
a tutorial (press \key{T} when asked for a species).

\para\para

\sex{The most important files}

\para

\begin{minipage}[t]{7cm}
The file \key{crawl.exe} in \crawl's main folder starts the game.
\\ \\
The \sex{settings/} directory contains \key{init.txt}, the options file for
\crawl\ (on linux systems there may also be a \key{.crawlrc} in your home
directory). Since the defaults are well suited for playing, you can ignore
these at first.
\\
This folder also contains \key{macro.txt}, a list of redefined key 
bindings and macros. These make playing \crawl\ even more convenient. 
You will probably not need to redefine key mappings until after you have
spent some time playing the game.
\end{minipage}
%
\spacecolumn
%
\begin{minipage}[t]{7cm}
The following files in the \sex{docs/} directory may be helpful:
\\
\key{crawl\_manual.txt} is the full manual. It explains all races, classes, 
item types etc. If you do not delight in manuals, you can put off the 
reading this file until later. You can browse the manual while playing 
(press \key{?}).
\\
\key{options\_guide.txt} describes all the options in minute detail. While
tweaking these can improve your \crawl\ experience, you will probably prefer
to skip this at first.
\\
\key{crawl\_macros.txt} explains macros and key bindings. You should probably
ignore this at first, too.
 % (Unless you experience serious problems with some keys, in which case should
 % look at the keymaps section.)
\end{minipage}

\para\para

\sex{Contact}

\para
	    
If you have questions concerning the game, or think you have found a bug, 
there are several places to contact the developers. First, there is the 
newsgroup \key{rec.games.roguelike.misc}. Since other games are 
discussed there as well, it is polite to flag your post with \key{-crawl-}.
If you are not familiar with Usenet netiquette, you might want to check 
that first, too.
Also try to maintain netiquette to the benefit of your addresses.
\\
Next, you are encouraged to file bug reports and feature requests on the 
\crawl\ homepage at \key{http://crawl-ref.sourceforge.net}. From there, you 
can also download the most recent version (both binaries or source, for 
Unix, Windows, OS X, and DOS).
\\
Finally, you can use the mailing list 
\key{crawl-ref-discuss@lists.sourceforge.net}
to discuss specific details of the game's design and implementation.

\para \para

\sex{Internet play}

\para

You can play \crawl\ online, both competing with other players and watching 
them. Check the homepage \key{crawl.akrasiac.org} for details, including 
information about additional servers. You just need a \texttt{ssh} or 
\texttt{telnet} console; on Windows, the \texttt{PuTTY} program works very 
well. Read \key{ssh\_guide.txt} in the \key{docs} folder for a step by step 
guide on how to set this up.

\para \para

\sex{Tiles}

\para

\crawl\ features an alternative to the classical ASCII display; Tile-based
Crawl is often easier accessible by new players. Tiles are available for 
Linux, Windows and OS X.
Unfortunately, it is not yet possible to combine tiles and internet play.
\end{document}
