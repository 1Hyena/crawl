\documentclass[a4paper,10pt]{article}

\usepackage{palatino}
\usepackage{mathpazo}           % optional [sc] for small caps
\usepackage[left=3cm,top=3cm,right=3cm,bottom=2cm,nohead,nofoot]{geometry}

\usepackage{graphicx}

\newcommand{\key}[1]{{{\texttt{\textbf{#1}}}}} % this does nasty things to underscores
\newcommand{\sex}[1]{{{\textbf{#1}}}} % \sec already defined


\newcommand{\crawl}{\textsc{Crawl}}
\newcommand{\dungeon}{\textsc{Dungeon}}
\newcommand{\soup}{\textsc{Stone Soup}}

\newcommand{\spacecolumn}{\begin{minipage}[t]{2cm}\phantom{xxxx}\end{minipage}}
\newcommand{\para}{\vspace{1.5ex}}
\setlength{\parindent}{0em}

\newcommand{\mc}[1]{\multicolumn{2}{l}{#1}}

\newcounter{abccounter}
\newenvironment{abcliste}{\begin{list}{(\alph{abccounter})}
                      {\usecounter{abccounter}
                       \setlength{\topsep}{0ex}
                       \setlength{\partopsep}{0ex}
                       \setlength{\listparindent}{0ex}
                       \setlength{\itemsep}{0ex}
                       \setlength{\parsep}{0ex}
                       \setlength{\leftmargin}{2em}
                       \setlength{\labelwidth}{2em}
                       \setlength{\parskip}{0ex}
                      }
                      }{\end{list}}

\pagestyle{empty}

\begin{document}


\begin{center}\textbf{\LARGE
\dungeon\ \crawl: Introduction, Files  and Contact
}\end{center}

This is the reference sheet for the roguelike game \dungeon\ \crawl,
specifically for the current version of the \soup\ branch. 
\crawl\ is a game of dungeon exploration, combat and magic, involving
characters of diverse skills, worshipping deities of great power and
caprice. To win, you'll need to be a master of tactics and strategy,
and prevail against overwhelming odds.

\para

Players of versions 0.3.4 and older beware: please read the file 
\key{key\_changes.pdf} in the \key{docs} directory for a list of
interface changes, and how you could possibly retrieve the 0.3.4
standards.

\subsection*{How to get started? (Information for new players)}

If you'd like to dive in immediately, your best bets are to either
start up a game and choose one of the tutorials (press \key{Ctrl-T} 
when asked for species), or print \key{quickstart.pdf} (in the 
\key{docs} directory). Studious readers might want to browse the 
manual (see below). Note that you can read both the quickstart text 
and the manual in-game; pressing \key{?} brings up a menu for that.

\para

\sex{Internet play}

You can play \crawl\ online, both competing with other players and watching 
them. Check the homepage \key{http://crawl.akrasiac.org} for details, including 
information about additional servers. You just need a \texttt{ssh} or 
\texttt{telnet} console; on Windows, the \key{PuTTY} program works very 
well. Read \key{ssh\_guide.txt} in the \key{docs} folder for a step by step 
guide on how to set this up.

\para 

\sex{Tiles}

\crawl\ features an alternative to the classical ASCII display; Tile-based
Crawl is often a lot more accessible by new players. Tiles are available for 
Linux, Windows and OS X.
Unfortunately, it is not yet possible to combine tiles and Internet play.

\subsection*{The most important files}

\begin{minipage}[t]{7cm}
The file \key{crawl.exe} (just \key{./crawl} if on Unix) in \crawl's main 
folder starts the game.
\\ \\
The \key{settings} directory contains \key{init.txt}, the options file for
\crawl\ (on Linux systems there may also be a \key{.crawlrc} in your home
directory). Since the defaults are well-suited for playing, you can ignore
these at first.
\\
This folder may also contain \key{macro.txt}, a list of redefined key 
bindings and macros. These make playing \crawl\ even more convenient. 
You will probably not need to redefine key mappings until after you have
spent some time playing the game.
\end{minipage}
%
\spacecolumn
%
\begin{minipage}[t]{7cm}
The following files in the \key{docs} directory may be helpful, all of
which can be read in-game (press \key{?}):
\\
\key{crawl\_manual.txt} is the full manual. It explains all species, jobs, 
item types etc. If you do not delight in manuals, you can put off the 
reading this file until later.
\\
\key{options\_guide.txt} describes all the options in minute detail. While
tweaking these can improve your \crawl\ experience, you will probably prefer
to skip this at first.
\\
\key{macros\_guide.txt} explains macros and key bindings. You should probably
ignore this at first, too.
 % (Unless you experience serious problems with some keys, in which case should
 % look at the keymaps section.)
\end{minipage}

\subsection*{Contact}
	    
If you have questions concerning the game, or think you have found a bug, 
there are several places to contact the developers. 

First, you are encouraged to file bug reports and feature requests on the 
\crawl\ homepage at \key{http://crawl-ref.sourceforge.net}. From there, you 
can also download the most recent version (both binaries or source, for 
Unix, Windows, OS X, and DOS).
\\
Next, there is the newsgroup \key{rec.games.roguelike.misc}. Since other 
games are discussed there as well, it is polite to flag your post with 
\key{-crawl-}. If you are not familiar with Usenet netiquette, you might 
want to check that first, too. Also try to maintain netiquette for the 
benefit of your addressees.
\\
Finally, you can use the mailing list 
\key{crawl-ref-discuss@lists.sourceforge.net}
to discuss specific details of the game's design and implementation.

\subsection*{License and history information}

This is a descendant of \textsc{Linley's} \dungeon\ \crawl. 
Development of the main branch stalled at version 4.0.0b26, with a final 
alpha of 4.1 being released by Brent Ross in 2005. Since 2006, the 
\dungeon\ \crawl\ \soup\ team has been continuing the 
development. See the \key{CREDITS} in the main folder for a myriad of 
contributors, past and present; \key{license.txt} contains the legal blurb.

\dungeon\ \crawl\ \soup\ is an open source, freeware roguelike. It is 
supported on Linux, Windows, OS X  and, to a lesser extent, on DOS. The 
source should compile and run on any reasonably modern Unix. \soup\ 
features both ASCII and graphical (Tiles) display.

\crawl\ gladly and gratuitously uses the following open source packages: \\
The \textbf{Lua} script language, see \key{docs/lualicense.txt}.\\
The \textbf{PCRE} library for regular expressions, see 
    \key{docs/pcre\_license.txt}.\\
The \textbf{Mersenne Twister} for random number generation, 
    \key{docs/mt19937.txt}.\\
The \textbf{SQLite} library as database engine; it is properly public domain.\\
% The \textbf{ReST} light markup language for the documentation.

\subsection*{How you can help}

If you like the game and you want to help making it better, there are a number 
of ways to do so:

\para

\textbf{Playtesting:}
At any time, there will be bugs --- playing and reporting these is a great 
help. There is a beta server at \key{http://crawl.develz.org} hosting the most 
recent version of the code. Besides pointing out bugs, you are also welcome to
bring up ideas on how to improve interface or gameplay. Bugs and new ideas can 
be added and discussed on the homepage, \key{http://crawl-ref.sourceforge.net}.

\para

\textbf{Vault making:}
Crawl uses many hand-drawn (but often randomised) maps. Making them is fun 
and easy. It's best to start with simple entry vaults (glance through 
\key{dat/entry.des} for a first impression). Later, you may want to read 
\key{docs/level\_design.txt} for the full power. If you're ambitious, new 
maps for branch ends are possible, as well.
If you've made some maps, you can test them on your system (no compiling 
needed) and then just mail them to the mailing list.

\para

\textbf{Speech:}
Monster talking provides a lot of flavour. Just like vaults, speech depends 
upon a large set of entries. Since most of the speech has been outsourced, 
you can add new prose. The syntax is effective, but slightly strange, so you 
may want to read \key{docs/monster\_speech.txt}.
Again, changing or adding speech is possible on your local game. If you 
have added something, send the files to the list.

\para

\textbf{Monster descriptions:}
You can look up the current monster descriptions in-game with \key{?/} or 
just read them in \key{dat/descript/monsters.txt}. The following conventions 
should be more or less obeyed: Descriptions ought to contain flavour text, 
ideally pointing out major weaknesses/strengths. No numbers, please. 
Citations are okay, but try to stay away from the most generic ones.
If you like, you can similarly modify the descriptions for features, items or
branches.

\para

\textbf{Tiles:}
Since version 0.4, tiles are integrated within \crawl. Having variants of 
often-used glyphs is always good. If you want to give this a shot, please 
contact us via the mailing list.

\para

\textbf{Patches:}
If you like to, you can download the source code and apply patches. Both 
patches for bug fixes as well as implementation of new features are very 
much welcome. Please be sure to read \key{docs/coding\_conventions.txt} first.

\para\para\para

Thank you, and have fun crawling!
\end{document}
